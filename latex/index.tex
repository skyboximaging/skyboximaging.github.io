\hypertarget{index_intro_sec}{}\section{Introduction}\label{index_intro_sec}
BusBoy is a framework that allows Java to run native processing algorithms on datasets. \hypertarget{index_basics}{}\section{Basics}\label{index_basics}
1. IMPLEMENT: Algorithms are implemented by extending the Algorithm class. The developer must implement the virtual method executeAlgorithm.

2. CONFIGURE: After instantiating the algorithm, developers should add the names of expected input and output datasets to their algorithm instances.

3. REGISTER: An instance of the algorithm should be registered with the AlgorithmRegistry. The easiest way to do this is via the macro BUSBOY\_\-REGISTER\_\-ALGORITHM. This will create a global variable that registers the algorithm when its library is loaded.

4. DEPLOY: The algorithm and its registration mechanism should be compiled into a shared library.\hypertarget{index_dataaccess}{}\section{Data Access}\label{index_dataaccess}
1. AlgorithmData: This is the root object of the data model. It provides access to the keyspace, input datasets, and empty output datasets.

2. Keyspace: The keyspace has an ordered list for each dimension of all the elements present in the input datasets. These lists can be iterated thru, e.g. to run an algorithm for each frame in the input data. All classes in the key api have \char`\"{}value\char`\"{} semantics (see \#5).

3. Adding Files: Algorithms may add new files to any of the output datasets. A DataFile object has \char`\"{}entity\char`\"{} semantics (see \#5). The dataset it is added to is resposible for eventually deleting it.

4. Adding Metadata: Algorithms may built up arbitrary metadata structures for each key. Any of the metadata types may be added to a dataset as the root object for a key. Building up a complex structure requires the use of MetadataList or MetadataMap as the root. Both of these classes have entity semantics. The remaining metadata classes have value semantics ( those with value semantics all extend CloneableMetadataElement).

5. Entity vs. Value semantics: The classes available for use by algorithm developers have two different kinds of semantics, \char`\"{}value\char`\"{} and \char`\"{}entity\char`\"{}: 